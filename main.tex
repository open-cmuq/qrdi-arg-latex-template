\documentclass[11pt]{qrdi-arg}

\newcommand{\PI}{}
\newcommand{\proposalNo}{ARG00-0000-000000}
\newcommand{\proposalTitle}{Project Title}

\newcommand{\draft}{}  % Remove for final version

\input{arg-macros}

\usepackage{tabularx,ragged2e,booktabs}
\newcolumntype{L}{>{\RaggedRight\arraybackslash}X}
\newcolumntype{b}{X}
\newcolumntype{s}{>{\hsize=.5\hsize}X}


%#############################################################################
%Document starts here

\begin{document}

\thispagestyle{empty}
\begin{center}
\textbf{\Large{ACADEMIC RESEARCH GRANT}}
\end{center}
\vspace{5mm}

%The first table in the ARG template
\begin{table}[h]
\setlength\tabcolsep{5pt}
\centering
\fontsize{11pt}{11pt}\selectfont
\renewcommand{\arraystretch}{3}
\noindent\begin{tabular}{|p{2.06in}| lll|}
\hline
\textbf{ARG ID}                             & \multicolumn{3}{l|}{\proposalNo}    \\ \hline
\textbf{Project title in English}           & \multicolumn{3}{p{4.71in}|}{\proposalTitle}    \\ \hline
\textbf{Project title in Arabic (optional)} & \multicolumn{3}{l|}{}    \\ \hline
\textbf{Type of Application}                & \multicolumn{3}{l|}{\radiobutton \hspace{0.01in}  New} \\                                                                                               & \multicolumn{3}{l|}{\radiobutton* Resubmission} \\ \hline

\textbf{Submitting Institution}             & \multicolumn{3}{l|}{Carnegie Mellon University in Qatar}    \\ \hline
\textbf{Lead PI (title, name, position)}    & \multicolumn{3}{l|}{}    \\ \hline 
\textbf{List of participants (PI name, collaborative institutions, PI residency)} & \multicolumn{3}{l|}{ 
\renewcommand{\arraystretch}{1.7}
\setlength\extrarowheight{3pt}

\begin{tabular}[t]{|p{1.3in}|p{1.3in}|p{1.3in}|}
\hline 
PI name & PI Institution & PI residency \\ \hline
        &                &              \\ \hline
\end{tabular} } \\[18ex] 
\hline
\textbf{Total funding requested} & \multicolumn{1}{l|}{X USD} & \multicolumn{1}{l|}{\makecell{\textbf{Project} \\\textbf{duration}}} & 36 months \\ 
\hline
\end{tabular}
\end{table}

%###########################################################################################
%###########################################################################################
%Add the table of contents to a new page in the template
\newpage
\tableofcontents

\instructions{
\begin{itemize}
  \item The research plan should describe the proposed research and
  its intended outcomes. It should cover the context, importance,
  methodology and all the associated work and activities you propose
  to carry out that will make this a high-quality research project.

  \item This document is the principal opportunity to explain to
  reviewers why your proposal should be  supported.
\end{itemize}

Before submitting this document:
\begin{itemize}
  \item Remove all the instructions in red.
  
  \item Update the table of contents (right click on table of
  contents-> Update Field-> Update the entire table).

  \item This document must not exceed 10 pages excluding cover page,
  table of contents and references, body of text in regular black
  Arial font size 11, single space and the margins as identified in
  the template (no less than 0.5 inches).

  \item In case of resubmission, the rebuttal section should not
  exceed 1.5 pages and is not included in the total page count.

  \item Appendices should be added under the ``Miscellaneous
  Document'' section on the system.

  \item Make sure that the information on the cover page matches the
  information entered on the system.
\end{itemize}

All sections are mandatory unless marked “if applicable”. Extra
sections should not be added. We recommend that you follow the
sub-sections indicated in the template. However, you are free to adapt
the structure of the sub-sections according to your  needs.
Applications that do not comply with the instructions may not be
accepted for review.
}
\newpage



%###########################################################################################
%###########################################################################################

\noindent 
\Part{Research Plan }
\setcounter{section}{-1}

\section{PROPOSAL SUMMARY}

\instructions{Please describe in up to 500 words the nature
of the proposed research project. This should be in plain English and
should cover the following items: What are the scientific objectives
of this project? What does your team seek to do and how will you do
it? What are the expected outcomes of the project? What is its
expected impact?}

%###########################################################################################
%###########################################################################################

\section{REBUTTAL (If Applicable)}
\label{sec:reb}

\instructions{If this proposal is a resubmission, include
here the questions \& responses to previous peer reviewers’ comments
and identify the substantive changes incorporated in the research
plan.  In addition, LPIs resubmitting proposals must point out in the
body of this research plan (e.g. in bold type, line in the margin,
underlining, italic, etc.) all revisions and modifications made in
response to the PRs’ comments.}

%###########################################################################################
%###########################################################################################

\section{INTRODUCTION}
\label{sec:intro}

%--------------------------------------------------------------------------------------------

\subsection{BACKGROUND AND PRELIMINARY DATA}
\label{sec:intro:back}

\instructions{Provide details on preliminary data related to
this proposal and on related research projects (ongoing or previous)
by members of the research team, including QNRF projects (if any).
Indicate any output (patents, publications, products, data sets, etc.)
obtained from those projects.}

%--------------------------------------------------------------------------------------------

\subsection{RESEARCH HYPOTHESIS AND OBJECTIVES}
\label{sec:intro:hyp&obj}

\instructions{State the objectives and hypothesis of your
proposal and your research approach. Explain the novelty, originality,
creativity, and, potentially transformative, aspects of your
proposal.}

%--------------------------------------------------------------------------------------------

\subsection{EXPECTED OUTCOME AND IMPACT OF THE PROJECT}
\label{sec:intro:out}

\instructions{Describe how the outcomes of this proposal
would create or advance knowledge and enhance understanding of
existing knowledge. Explain to what extent your proposal will support
the RDI national priority areas
(\href{Link}{https://qrdi.org.qa/en-us/Strategic-Focus}) and educate
the next generation of scientists.}

%###########################################################################################
%###########################################################################################

\section{WORK PLAN}
\label{sec:workplan}

%--------------------------------------------------------------------------------------------

\subsection{METHODOLOGY}
\label{sec:workplan:meth}

\instructions{Clearly describe and explain your methodology,
including techniques, procedures, and tools to be used.}

%------------------------------------------------------------------------------------------

\subsection{TECHNICAL DESCRIPTION BY WORK PACKAGE}
\label{sec:workplan:wps}

\instructions{Please enter your WPs details as per the table
below. Repeat for each WP.}

%------------------------------------------------------------------------------------------
\subsubsection{\underline{Work Package (WP) \textcolor{red}{\#} :}}
%------------------------------------------------------------------------------------------
\vspace{-5mm}
\begin{table}[H]
\begin{tabularx}{\textwidth}{@{}|b|s|s|@{}}
\hline
{\underline{\textbf{Work Package Title}}}                    & {\underline{\textbf{Start Month}}} & { \underline{\textbf{End Month}}} \\ 
\hline
\instructions{Enter the work package title here}  &                                    &                                   \\ 
\hline
\end{tabularx}
\end{table}
\vspace{-5mm}

\begin{table}[H]
\begin{tabularx}{\textwidth}{|l|@{}|L|L|L|L|@{}}
\hline
\underline{\textbf{Participant Name}} &
  \textbf{\begin{tabular}[c]{@{}l@{}}\underline{Role (KI, PDF,}\\ \underline{RA, etc.)}\end{tabular}} &
  \textbf{\begin{tabular}[c]{@{}l@{}}\underline{Effort Inside} \\ \underline{Qatar}\end{tabular}} &
  \textbf{\begin{tabular}[c]{@{}l@{}}\underline{Effort Outside} \\ \underline{Qatar}\end{tabular}} &
  \textbf{\begin{tabular}[c]{@{}l@{}}\underline{Total Effort} \\ \underline{in Days}\end{tabular}} \\ 
\hline
\begin{tabular}[c]{@{}l@{}}\instructions{Participant Name 1 (leader of this WP)}\end{tabular} &    &    &    &   \\ 
\hline
\begin{tabular}[c]{@{}l@{}}\instructions{Participant Name 2}\end{tabular} &  &  &  & \\ 
\hline
\begin{tabular}[c]{@{}l@{}}\instructions{Participant Name 3}\end{tabular} &  &  &  & \\ 
\hline
\end{tabularx}
\end{table}
\vspace{-5mm}

\begin{table}[H]
\begin{tabular}{|p{\textwidth}|}
\hline
\textbf{Tasks and  Deliverables of the WP} \\ 
\hline
\instructions{
  Task 1.1:

  Objective, methodology, deliverables, and duration

  Task 1.2:

  Objective, methodology, deliverables, and duration } \\
\hline
\end{tabular}
\end{table}

\vspace{-5mm}
\begin{table}[H]
\begin{tabularx}{\textwidth}{@{}|L|@{}}
\hline
\textbf{Performance Site(s)}                                                 \\ \hline
\instructions{Enter the performance site(s) here (if applicable)} \\ 
\hline
\end{tabularx}
\end{table}

%------------------------------------------------------------------------------------------

\subsection{PROJECT TIMELINE}
\label{sec:workplan:timeline}
\noindent

\instructions{Indicate in the table below the timeline for
each WP and task(T) as per the previous section. Highlight cells of
the month(s) and year(s) of each WP and associated tasks (T).}
\vspace{-0.2in}
\input{timeline}


%###########################################################################################
%###########################################################################################

\section{PROJECT MANAGEMENT}
\label{sec:management}
\noindent

\instructions{Describe how the project is managed and co-ordinated among the various collaborators. 
Identify the main risks and outline mitigation plans. Risks include
threats to health and safety to researchers and subjects of research,
delays, for example, in recruitment (researchers, students, patients,
etc.) or procurement, or in obtaining regulatory (e.g., ethical)
permissions. Risks could also be scientific, such as degrees of
certainty, long execution times of codes or malfunction of equipment
and facilities. You should describe the nature and degree of risk, its
likelihood of occurrence, and outline plans of mitigating it.}

%------------------------------------------------------------------------------------------

\subsection{SYNERGY AND RELEVANCE OF THE RESEARCH TEAM}
\label{sec:management:synergy}
\noindent

\instructions{Show how team members complement each other’s
expertise and describe the added value of the collaborations, with a
special emphasis on the contribution by the LPI and his/her team (KIs,
unnamed roles, consultants). Interdisciplinarity and the inclusion of
specific research teams must be justified in accordance with the
project objectives.}

%------------------------------------------------------------------------------------------

\subsection{ETHICAL AND REGULATORY REQUIREMENTS}
\label{sec:management:ethical}
\noindent

\instructions{This section is required for all proposals
involving research on human subjects, research animals, biohazardous
materials and/or recombinant nucleic acids. Applicants must provide
sufficient information here to allow the reviewers to determine if the
involvement of human subjects and/or if the use of research animals
and/or biohazardous materials and/or nucleic acids are appropriate. In
addition, applicants must demonstrate how the proposal takes the
appropriate measures to ensure the protection of human subjects and/or
ensures the appropriate care and use of research animals as
applicable. The proposal must meet all the requirements set by the
applicable Ministry of Public Health policies for the protection of
human subjects, care and use of research animals, and/or use of
biohazardous materials and/or nucleic acids which can be found here.
Applicants are encouraged to consult with their institution’s research
compliance office before submitting their proposals. }

%------------------------------------------------------------------------------------------

\subsection{DISSEMINATION AND IMPLEMENTATION OF RESULTS}
\label{sec:management:diss&impl}
\noindent

\instructions{Describe plans to disseminate and, where
applicable, implement the project’s results and outcomes.}

%------------------------------------------------------------------------------------------
%------------------------------------------------------------------------------------------
\newpage
\section{REFERENCES}
\noindent
\bibliographystyle{unsrt}
\bibliography{refs.bib}

\instructions{Include the list of bibliographic references used in the research plan.}

\end{document}

